% TEMPLATE for Usenix papers, specifically to meet requirements of
%  USENIX '05

% See http://static.usenix.org/events/samples/template.la for examples.

% Note that since 2010, USENIX does not require endnotes. If you want
% foot of page notes, don't include the endnotes package in the 
% usepackage command, below.

%
% This is a pre-publication paper which we intend to submit to FOCI-14
% While the FOCI14 event has no home page, the previous one was:
%   https://www.usenix.org/conference/foci13
%

\documentclass[letterpaper,twocolumn,10pt]{article}
\usepackage{usenix,epsfig,amssymb,amsmath,url}
\begin{document}
\date{}

\title{
\Large \bf Going Dark: Phrase Automated Nym Discovery Authentication\\
\small or,\\
Finding friends and lovers using the PANDA protocol to re-nymber after everything is lost\\
\small or, \\
discovering new Pond users easily
}

\author{
{\rm Jacob Appelbaum}\\
\and
{\rm another cypherpunk}\\
}

\maketitle

% Use the following at camera-ready time to suppress page numbers.
% Comment it out when you first submit the paper for review.
\thispagestyle{empty}

\subsection*{Abstract}

We consider the problem of two parties seeking to establish secure
communications given {\it only} a shared secret.

Communication between two parties relies on in-band and out-of-band methods
when addressing identifiers change or when they are not yet known to the
interested parties. We summarize similar problems and present a brief overview
of previous attempts at solving the presented problems. Finally we propose a
system, which we have implemented, by which a pair or a group of people or
machines may derive new identifiers using only a password or shared key. Thus
we remove the requirement that any other communication must occur before
re-establishing entirely new identifiers.

\section{Introduction}

% we should rephrase this to not be so closely worded to the Abstract
We propose a system by which a pair or a group of people or machines may derive
new identifiers using only a password or shared key in a forward secret manner.
Additionally, we propose using this shared secret as the basis for
authentication, key exchange and as a discovery process for other identifiers.
We have implemented the described system and describe the specifics of our
implementation.

\section{The context of modern communication systems}
\label{sec:context}

Modern communication systems are designed with intentional flaws such as
CALEA\cite{CALEA} and with an adversaries that have budgets in the tens of
billions of US dollars\cite{NSABUDGET}. Adversaries such as the NSA
specifically attempt to sabatoge cryptographic projects\cite{BULLRUN} used in
the creation of secure communications systems. The current global
communications network allow for dragnet surveillance based on so-called {\it
selectors} and even world leaders\cite{MERKELPHONE} are not exempt from such
surveillance. Static identifiers such as telephone numbers, email addresses,
hardware serial numbers and other unique identifiers ensure linkability across
vast datasets as the core of selector based surveillance.
% XXX: should we cite something about selector based surveillance? Perhaps?

\section{An overview of similar problems and their respective solutions}

Users face many similar discovery or re-discovery problems in common daily
life. Usually these problems are resolved with a hybrid of social and
centralized solutions where a third party is not treated as an adversary.

\subsection{The telephone re-numbering problem}
\label{renumber:telephone}

Telephone numbers change for various reasons. After a new number or a new
device with a new number is issued, a kind of discovery process is started
whereby the party may start to use the new number exclusively. It is also
possible that the old number will link to the new number or only release the
new number as it has superseded the original number.

Telephone numbers are generally not user selected but are rather issued by a
central authority; so called vanity numbers may be selected for a fee from
some operators.

The process of renumbering a telephone is generally solved in five different
ways. The first may come in the form of a message provided by the operator
that discloses the new number and disconnects the caller after an automated
message. The second is socially or through a publication of some kind;
generally when two people meet a new number may be disclosed or when a known
distribution point such as a newspaper begins to print new numbers. The third
is through a party reading the caller information from their received call or
SMS list or by reviewing their bill. The fourth is through
misclosure\cite{kellycaine}, perhaps by a third party who may or may not be
authorized to disclose this information. The fifth is through a central listing
service of some kind, such as a phonebook, published by a network operator.

\subsection{The social nickname re-naming problem}
\label{rename:social-nickname}

People generally have full legal names and names by which they are known
socially. These full legal name mappings change through various legal means
that are out of scope. Nicknames are commonly held names that otherwise have
little or no legal status - when a party changes their nickname, they suffer
from the same issues faced by telephone renumbering\ref{renumber:telephone}.
Unlike telephone re-numbering, generally a person or group may pick any name
they choose as a new nickname. Such new nicknames may collide and cause
confusion though largely this may be irrelevant for a social group that has
other social identifiers.

\subsection{The nym renyming problem}

Electronic identities generally suffer from the same issues as telephone
renumbering\ref{renumber:telephone} and re-naming\ref{rename:social-nickname}
with the additional complexity of not having a central authority or out-of-band
communications methods by which to reconnect.

\subsection{The problem created by data loss}
\label{sec:problem}

Systems are often stolen or compromised in a way that results in total
dataloss for the user. In some cases the data may be compromised after a
variable window of time and in such a window, automatically changing
indentifiers may prevent full and on-going compromise of a pair-wise
communications channel.

\section{Gently Rockin' Zooko's Triangle}
\label{sec:zooko}

Zooko's Triangle\cite{zookostriangle} is generalized as having the choice
between any two of the following three properties when designing a naming
system: secure, decentralized and human-meaningful. In the original Domain Name
System, a domain name is decentralized and human-meaningful. In
Pond\cite{pond}, the addressing system properties are secure and decentralized.

We propose that, for a brief window of time, users should have a way to
change the chosen properties of the naming system.  We assert that trading the
secure property for the human-meaningful property in the form of phrases may be
a useful trade-off in a time limited, forward secure fashion.

Furthermore, we note that Zooko's Triangle is missing an important qualifier,
which is the passing of time and its relationship with each of the properties.

\section{PANDA}

The PANDA protocol uses EKE2, Diffie-Hellman, a pre-arranged meeting point and
a slow memory intensive key derivation process.

\subsection{EKE2}

EKE2 is very similar to SPAKE2 but uses an ideal cipher to encrypt the
Diffie-Hellman values rather than masking them with secret group elements. EKE2
is used to go from a shared-secret to an ephemeral shared-key and then the
KeyExchange messages are encrypted with that key and exchanged via the second
meeting ID.

\subsection{Diffie-Hellman}

Our Diffie-Hellman group of choice is curve25519\cite{bernstein2006curve25519}
which results in a 255-bit Diffie-Hellman value, which is larger than the
128-bit block size of most ciphers.

\subsection{Password Authenticated Key Exchange}

This problem falls into the topic of Password Authenticated Key Exchange
(PAKE), which is a well studied problem. The first PAKE, Encrypted Key Exchange
(EKE), was described by Bellovin and Merritt\cite{bellovin1992encrypted} in
1992. However, we have one additional requirement: that the protocol execute in
a single round. We define a round as a message sent by each party
simultaneously, thus a message in a given round cannot depend on any other
message in the same round.  We are imposing this requirement as our meeting
point may have very high latency and we have no way of assigning the labels
`client' and `server' to our parties.  Many PAKE protocols implicitly assume
that the parties can behave differently based on these labels (including the
original EKE), but our parties have {\it only} a shared secret. An exchange of
messages could be used to break this symmetry but the additional latency is
undesirable.

The authors are most familiar with Simple Password-Based Encrypted Key Exchange
Protocols number two (SPAKE2)\cite{abdalla2005simple}, but SPAKE2 appears to be
unsuitable because it masks Diffie-Hellman values with a mask value raised to
the power of the password and requires different mask values for the two
parties. Due to our symmetry requirement the two parties would have to use the
same value mask. We have proven that the chosen computational Diffie Hellman
(CCDH) problem from the SPAKE paper is equally strong when the two mask values
are equal, however the mask values appear again in the security proof and it's
not obvious how to patch that second hole.
% XXX: This last sentence could use some rewording to make it clearer

We have chosen to use Encrypted Key Exchange two
(EKE2)\cite{bellare2000authenticated} as our PAKE. EKE2 is very similar to
SPAKE2 but uses an ideal cipher to encrypt the Diffie-Hellman values rather
then masking them with secret group elements.

Using a typical block cipher mode would
result in an attacker having a discomforting amount of control of the resulting
plaintext. Using an Authenticated Encryption with Associated Data (AEAD) mode would give an attacker an authenticator which
they could use to mount an offline, brute-force attack on the shared
secret\footnote{The attacker may already be able to mount an offline,
brute-force attack on the shared secret by using an identifier revealed to the
meeting point. But we still don't want to give them another avenue for this
attack in case a particular meeting point doesn't require that}. A wide-block
construction like LIONESS\cite{anderson1996two} doesn't appear to be
applicable because the 256-bit block is too small to split into usefully sized
left and right parts for the Luby-Rackoff construction. Many other wide-block
constructions have subsequently been proposed including
EME\cite{halevi2004parallelizable}, CMC\cite{halevi2004parallelizable},
XCB\cite{mcgrew2004extended}, HCTR\cite{wang2005hctr} and others. However, the
area is a patent minefield.

We instantiate the ideal-cipher for EKE2 with a 256-bit block
Rijndael\cite{daemen2002design}. Although AES specified only a 128-bit block
size, Rijndael was originally specified with 128, 192 and 256-bit block sizes.
It's very uncommon to find an Rijndael implementation that implements a 256-bit
block size but since performance isn't a concern and nor are side-channels, a
custom, toy Rijndael implementation is sufficient.

\subsection{Meeting points}

We propose agreeing on a meeting point, such as an address or name of a
centralized HTTP server, one hosted on a Tor Hidden
Service\cite{Dingledine04tor:the}, or the use of a
decentralized key value store system.

Once we have established a meeting point, the two parties need to bootstrap
their shared secret into a cryptographic key exchange and a more efficient
method of communicating in the future.\footnote{For example, the parties may wish to
exchange OpenPGP public keys and email addresses in a confidential and
authentic fashion. As a further example, Pond users may discover the
keys for their peers with such a system.}
% add \cite for OpenPGP

\subsection{Key and meeting point derivation}

Each PANDA user computes a tag and a key as part of a key derivation process.
This process utilizes scrypt\cite{scrypt} to slow brute force or
even well chosen dictionary attacks.
% XXX: We should describe this process in more detail

\section{Key exchange details in practice with Pond}

Pond supports two methods of key exchange. The first is manual: the
KeyExchange messages are serialised with Privacy-enhanced Electronic Mail (PEM)\cite{pem} and the user is expected to
exchange them with the intended contact, authentically and confidentially. For
example, a pre-existing relationship with Pretty Good Privacy (PGP)\cite{pgp} or
Off-The-Record (OTR)\cite{otr} keys could be used to bootstrap Pond.

We implemented\cite{panda} the PANDA discovery technique for the
Pond\cite{pond} messaging system as the second method of key exchange. PANDA
allows a relationship to be established via a shared-secret. This works by
using a {\it meeting point} which, currently, involves contacting a central
HTTPS server over the Tor\cite{Dingledine04tor:the} network. The shared secret is processed with
scrypt\cite{scrypt} into a key and two {\it meeting} IDs. The meeting point
simply acts as a pinboard where messages can be posted to a given meeting ID.
If a different message has already been posted to the same ID then it's
returned. The meeting point only allows two different messages to exist for a
given meeting ID until it's garbage collected (in about a week).

\subsection{Pond key exchange details}

The Pond key exchange between two users involves each creating a KeyExchange
protobuf\cite{pondprotobufs}. Since Pond servers demand a group signature
before accepting a message for delivery, there are no public Pond addresses
that can be used to bootstrap a Pond relationship.  Rather we currently assume
an external, confidential and authentic means for users to exchange KeyExchange
messages.

A Pond KeyExchange message contains a group member private key (which is why
they need to be confidential), as well as the information needed to direct a
message to a user: their home server address and the user's public identity at
that server.  They also contain an Ed25519 key and the intent is that, in the
future, users will be able to update their home servers by sending messages to
each of their contacts. Group member revocations will also be signed by this
key.

\subsection{Adding entropy to a human-meaningfully memorable phrase}

The shared secret itself can consist of a passphrase, a time, and a deck or two
of cards. Since humans are poor at generating and remembering secrets, a deck
of cards, shuffled and split in half, serves quite well. Decks of cards are also
commonly available and don't have the downsides that bringing a mobile device
to a meeting might have.

Although the two halves of the deck are different, they are sufficient to
define a common, canonical set of cards. For Pond, the cards are ordered and
the canonical half is the one with the majority of the lowest numbered card. If
both halves have the same number of the lowest numbered card then the next
lowest is considered. If each half has exactly the same cards then both are
canonical.

A single deck of cards results in an approximately 49 bit shared secret and two
decks is 100 bits. The stacks of cards are a problem to dispose of (unlike a
memory, they can be obtained by dumpster diving) but they only need to be
protected until the key exchange is complete. After that they are no longer
sensitive.

While we haven't implemented further entropy sources, we believe that many
common household objects may also serve as a time limited entropy source. As an
example, additional entropy could be added from other sources such as the state
of a Rubik's cube or a similar object.

\section{Conclusion}

Our design allows a pair or a group of people to discover or rediscover
each-other with knowledge of only a shared secret or human memorable
passphrase with optional entropy sources. Additionally, we believe that it
provides forward secrecy against attackers who wish to correlate names even
after a full client compromise.

\section*{Acknowledgements}

We would like to thank Leif Ryge, Steve Weis, Ralf Philip-Weinmann, Julian
Assange and other anonymous cypherpunks who contributed valuable feedback in
the creation of this design.

{\footnotesize \bibliographystyle{acm}
\bibliography{bibliography}}

\end{document}
